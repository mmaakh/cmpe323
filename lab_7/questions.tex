\documentclass{article}
\usepackage{listings}
\usepackage{color}
\definecolor{mygreen}{rgb}{0,0.6,0}
\definecolor{mygray}{rgb}{0.5,0.5,0.5}
\definecolor{mymauve}{rgb}{0.58,0,0.82}

\lstset{ %
  backgroundcolor=\color{white},   % choose the background color; you must add \usepackage{color} or \usepackage{xcolor}
  basicstyle=\footnotesize,        % the size of the fonts that are used for the code
  breakatwhitespace=false,         % sets if automatic breaks should only happen at whitespace
  breaklines=true,                 % sets automatic line breaking
  captionpos=b,                    % sets the caption-position to bottom
  commentstyle=\color{mygreen},    % comment style
  deletekeywords={...},            % if you want to delete keywords from the given language
  escapeinside={\%*}{*)},          % if you want to add LaTeX within your code
  extendedchars=true,              % lets you use non-ASCII characters; for 8-bits encodings only, does not work with UTF-8
  frame=single,                    % adds a frame around the code
  keepspaces=true,                 % keeps spaces in text, useful for keeping indentation of code (possibly needs columns=flexible)
  keywordstyle=\color{blue},       % keyword style
  morekeywords={*,...},            % if you want to add more keywords to the set
  numbers=left,                    % where to put the line-numbers; possible values are (none, left, right)
  numbersep=5pt,                   % how far the line-numbers are from the code
  numberstyle=\tiny\color{mygray}, % the style that is used for the line-numbers
  rulecolor=\color{black},         % if not set, the frame-color may be changed on line-breaks within not-black text (e.g. comments (green here))
  showspaces=false,                % show spaces everywhere adding particular underscores; it overrides 'showstringspaces'
  showstringspaces=false,          % underline spaces within strings only
  showtabs=false,                  % show tabs within strings adding particular underscores
  stepnumber=1,                    % the step between two line-numbers. If it's 1, each line will be numbered
  stringstyle=\color{mymauve},     % string literal style
  tabsize=2,                       % sets default tabsize to 2 spaces
}
\title{Post-lab Questions of Lab 7}
\author{CMPE 324}
\begin{document}
    \maketitle
    \section{Question 1}
        Consider the scenario where you are about to implement a TCP server
        (e.g. a HTTP server). From the perspective of socket programming, your
        code should call the following functions (sorted randomly):
        \begin{itemize}
            \item \texttt{accept}.
            \item \texttt{read} and \texttt{write}.
            \item \texttt{socket}.
            \item \texttt{close}.
            \item \texttt{bind}.
            \item \texttt{listen}.
        \end{itemize}

        Your task in this question is to sort the functions above in the
        \emph{right order}, that allows the program to (ultimately) read/write
        from/to connected peers.

    \section{Question 2}
        From the POSIX functions above, list those that block\footnote{Sockets
        can be in blocking, or nonblocking modes. In nonblocking mode, none of
        the functions above block. In blocking mode, some of them block. In lab
        7, we worked on blocking sockets only, and this question is specific to
        blocking sockets.}.

    \newpage
    \appendix
    \section{Function synopsis}
        This appenedix lists relevant function synopsis for your own reference.
        The order by which the functions are listed here is also random (intentional).
        \begin{figure}[tbh]
            \centering
            \lstinputlisting[language=C]{./html/tcpaccept}
            \caption{The synopsis of the \texttt{accept} function.}
            \label{fig:accept}
        \end{figure}
        \begin{figure}[tbh]
            \centering
            \lstinputlisting[language=C]{./html/tcpread}
            \caption{The synopsis of the \texttt{read} function.}
            \label{fig:read}
        \end{figure}
        \begin{figure}[tbh]
            \centering
            \lstinputlisting[language=C]{./html/tcpwrite}
            \caption{The synopsis of the \texttt{write} function.}
            \label{fig:write}
        \end{figure}
        \begin{figure}[tbh]
            \centering
            \lstinputlisting[language=C]{./html/tcpsocket}
            \caption{The synopsis of the \texttt{socket} function.}
            \label{fig:socket}
        \end{figure}
        \begin{figure}[tbh]
            \centering
            \lstinputlisting[language=C]{./html/tcpclose}
            \caption{The synopsis of the \texttt{close} function.}
            \label{fig:close}
        \end{figure}
        \begin{figure}[tbh]
            \centering
            \lstinputlisting[language=C]{./html/tcpbind}
            \caption{The synopsis of the \texttt{bind} function.}
            \label{fig:bind}
        \end{figure}
        \begin{figure}[tbh]
            \centering
            \lstinputlisting[language=C]{./html/tcplisten}
            \caption{The synopsis of the \texttt{listen} function.}
            \label{fig:listen}
        \end{figure}

\end{document}
